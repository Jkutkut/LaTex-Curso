\documentclass[11]{article} % Documento tipo articulo, con fuente por defecto de 11pt
\usepackage[utf8]{inputenc} % Uso de carateres utf-8
\usepackage[T1]{fontenc} % Usa fuentes "ricas", que facilitan el copy-paste en la salida
\usepackage{xcolor}
\usepackage{float}

% para meter ecuaciones
\usepackage{amsfonts}
\usepackage{mathtools}

\usepackage[spanish]{babel} %cambia todos los textos auto al español
\usepackage{svg}


\title{Bloque1-primera prueba}
\author{jorgeregonzalezetsit }
\date{May 2020}

\begin{document}

\maketitle

% section hace trozos que llamas como quieras
\section{Introducción}
cosas que cuento en la introducción

\section{Segunda parte}
Se van numerando de manera automática

%Modificar el texto:
\textbf{texto en negrita} \textit{texto en cursiva} \textsl{suave}
\uppercase{ohhoafh} \texttt{kfjañlfajñklfjlñ}

% Bloques o grupos
% las llaves {} son como () en matemáticas

{\small Texto con llaves que contiene las propiedades que hemos puesto}

%tamaño de fuente personalizado:
{\fontsize{56}{60}\selectfont personalizado}

%Cambio de color:
% no vienen por defecto en latex (xcolor package):

puedes colorear lo que quieras,{\color{red}\huge hola en color rojo}, ya que está entre llaves. También {\colorbox{orange}{texto que uso}}

%Crear colores
\definecolor{rojo_guay}{RGB}{255,100,0}
{\color{rojo_guay} jofajfoa}


% Entornos:
% Tiene distintos commandos, normas...
% para hacer listas...

%init entorno
\begin{itemize}
    \item primer elemento
    \item segundo
    \item tercero
\end{itemize}

%Lista ordenada:
{\color{rojo_guay}
\begin{enumerate}
    \item primer
    \item segun
    \item tercer
\end{enumerate}
}

% Listas anidadas:
\begin{enumerate}
    \item tiene 2 partes:
    \begin{itemize}
        \item 1
        \item 2
        \item ...
    \end{itemize}
    \item otra
\end{enumerate}

%Salto de línea
dejar una línea en blanco

sigo aquí

%forzar salto de página:
\newpage

% ------------------------------------------------------

\section{\huge Ecuaciones:}

%ecuaciones en línea : usando $$
Pitágoras: $a^2 +b ^2 = c^ 2$

%bloques de ecuaciones: Usando entornos
\begin{equation}
% Subindice: _
% Super indice: ^
    \sum_{k=1}^{\infty} = 1 + \frac{a + e}{f-5} + \sqrt{urn}
\end{equation}

\begin{equation}
    %integral de contorno con \oint
    \oint_{c}
\end{equation}

\begin{equation}
    \int_a^b \vec B \cdot d \vec l + \partial l
\end{equation}

\subsection{El entorno \texttt{gather}:}


% el * es para que no salga la referencia
\begin{gather*}
    a + b + c
    esto está en la misma línea
\end{gather*}

\subsection{entorno \texttt{aling}}

% ahora el * centra el texto
\begin{align*}
    b &= \mathdf{hola}\\
    a &= otra cosa
\end{align*}


\subsection{Matrices}

%usamos otro entorno:

\begin{equation}
    \label{ecuacion a}
    \begin{bmatrix}
    a & b \\
    c & d 
    \end{bmatrix}
\end{equation}

\newpage
\section{Imagenes:}
% Latex crea las cosas de manera vectorial, por tanto puedes hacer el zoom que quieras

\begin{figure}[H]
%Ojo, la planta donde le apetece. Para arreglarlo, usar [H]
    \centering
    \includegraphics[height=3cm]{LOGOTIPO leyenda color JPG p.png}
    \caption{Caption}
    \label{fig:logo de la ETSIT}
\end{figure}

\begin{figure}[H]
%Ojo, la planta donde le apetece. Para arreglarlo, usar [H]
    \centering
    % ajuste porcentual
    \includegraphics[width=0.75\textwidth]{LOGOTIPO leyenda color JPG p.png}
    \caption{Caption}
    \label{fig:logo de la ETSIT}
\end{figure}

\begin{figure}[H]
%Ojo, la planta donde le apetece. Para arreglarlo, usar [H]
    \centering
    % ajuste porcentual
    \includegraphics[width=0.75\textwidth,trim{4cm,0cm,0cm,0cm}]{LOGOTIPO leyenda color JPG p.png}
    \caption{Caption}
    \label{fig:logo de la ETSIT}
\end{figure}

% OJOOOOO! .svg no está soportado de manera nativa, usar paquete

\section{referencias internas}
\label{section: cosa}

en la \ref{ecuacion a} ecuacion 1 hemos puesto la cosa hdalkfhallkjfal. Como igual cambiamos estas ecuaciones de orden, usamos el command label

otra referencia \ref{section: cosa}

\end{document}
